\documentclass[12pt]{article}
\usepackage[a4paper,margin=1in]{geometry}
\usepackage{fancyhdr}
\usepackage{setspace}
\usepackage{courier}
\usepackage{verbatim}
\usepackage{titlesec}
\usepackage{fvextra}
\usepackage[T1]{fontenc}
\usepackage[utf8]{inputenc}

\usepackage{listings}
\usepackage{xcolor}



\pagestyle{fancy}
\fancyhf{}
\fancyfoot[C]{\thepage}

\setlength{\parindent}{0pt}
\setstretch{1.1}

\titleformat{\section}{\large\bfseries}{}{0pt}{}

\lstdefinestyle{pythonstyle}{
    language=Python,
    basicstyle=\ttfamily\small,
    keywordstyle=\color{blue}\bfseries,
    commentstyle=\color{green!50!black},
    stringstyle=\color{red!60!black},
    numbers=left,
    numberstyle=\tiny,
    breaklines=true,
    %breakanywhere=true,
    showstringspaces=false,
    frame=single
}

\begin{document}


\begin{titlepage}
\centering

\vspace*{1cm}

{\Large \textbf{CENTRAL BOARD OF SECONDARY EDUCATION}}\\[0.5cm]
{\large \textbf{ACADEMIC SESSION 2025–2026}}\\[1.5cm]

{\Huge \textbf{COMPUTER SCIENCE}}\\[0.3cm]
{\Large (083)}\\[1.5cm]

{\LARGE \textbf{PRACTICAL FILE}}\\[2cm]

\begin{tabular}{p{6cm} p{8cm}}
\textbf{Student Name}      & : \underline{\hspace{7cm}} \\[0.6cm]
\textbf{Class / Section}   & : XII \underline{\hspace{5.5cm}} \\[0.6cm]
\textbf{Roll Number}       & : \underline{\hspace{7cm}} \\[0.6cm]
\textbf{Student ID}        & : \underline{\hspace{7cm}} \\[0.6cm]
\textbf{School Name}       & : \underline{\hspace{7cm}} \\[0.6cm]
\textbf{Subject Teacher}   & : \underline{\hspace{7cm}} \\
\end{tabular}

\vfill

%\begin{center}
%\textbf{}
%\end{center}

\vspace{1cm}

\end{titlepage}


\newpage
\vspace{1em}

\section*{Date : \hfill Experiment No: 1}

\textbf{Program 1: Input any number from user and calculate factorial of a number}


%\begin{verbatim}
\begin{lstlisting}[style=pythonstyle]
# Program to calculate factorial of entered number
num = int(input("Enter any number :"))
fact = 1
n = num             # storing num in n for printing
while num>1:        # loop to iterate from n to 2

 fact = fact * num
 num-=1

print("Factorial of ", n , " is :",fact)

\end{lstlisting}

\textbf{OUTPUT}
\begin{verbatim}
Enter any number :6
Factorial of  6  is : 720
\end{verbatim}



\newpage
\section*{Date : \hfill Experiment No: 2}

\textbf{Program 1: Input any number from the user and check if it is Prime no. or not}

%\begin{verbatim}
\begin{lstlisting}[style=pythonstyle]
#Program to input any number from user
#Check it is Prime number or not
import math
num = int(input("Enter any number :"))
isPrime=True
for i in range(2,int(math.sqrt(num))+1):
          if num % i == 0:
                    isPrime=False

if isPrime:
          print("## Number is Prime ##")
else:
          print("## Number is not Prime ##")
\end{lstlisting}

\textbf{OUTPUT}
\begin{verbatim}
Enter any number :117
## Number is not Prime ##

>>>
Enter any number :119

## Number is not Prime ##
>>>
Enter any number :113
## Number is Prime ##
>>>
Enter any number :7

## Number is Prime ##
>>>
Enter any number :19

## Number is Prime ##
\end{verbatim}



\newpage
\section*{Date : \hfill Experiment No: 3}

\textbf{Program : Write a program to find sum of elements of List recursively}

\begin{lstlisting}[style=pythonstyle]
#Program to find sum of elements of list recursively
def findSum(lst,num):
          if num==0:
                    return 0
          else:
                    return lst[num-1]+findSum(lst,num-1)


mylist = []                   # Empty List
#Loop to input in list
num = int(input("Enter how many number :"))
for i in range(num):
          n = int(input("Enter Element "+str(i+1)+":"))
          mylist.append(n)    #Adding number to list

sum = findSum(mylist,len(mylist))
print("Sum of List items ",mylist, " is :",sum)
\end{lstlisting}

\textbf{OUTPUT}
\begin{verbatim}
Enter how many number :6
Enter Element 1:10
Enter Element 2:20
Enter Element 3:30
Enter Element 4:40

Enter Element 5:50
Enter Element 6:60
Sum of List items  [10, 20, 30, 40, 50, 60]  is : 210
\end{verbatim}



\newpage
\section*{Date : \hfill Experiment No: 4}

\textbf{Program 1: Write a program to calculate the nth term of Fibonacci series}

%\begin{verbatim}
\begin{lstlisting}[style=pythonstyle]
#Program to find 'n'th term of fibonacci series
#Fibonacci series : 0,1,1,2,3,5,8,13,21,34,55,89,...
#nth term will be counted from 1 not 0

def nthfiboterm(n):
if n<=1:

return n
else:

return (nthfiboterm(n-1)+nthfiboterm(n-2))

num = int(input("Enter the 'n' term to find in fibonacci :"))
term =nthfiboterm(num)
print(num,"th term of fibonacci series is :",term)
\end{lstlisting}

\textbf{OUTPUT}
\begin{verbatim}
Enter the 'n' term to find in fibonacci :10
10 th term of fibonacci series is : 55
\end{verbatim}



\newpage
\section*{Date : \hfill Experiment No: 5}

\textbf{Program : Program to search any word in given string/sentence}

%\begin{verbatim}
\begin{lstlisting}[style=pythonstyle]
#Program to find the occurence of any word in a string
def countWord(str1,word):
          s = str1.split()
          count=0
          for w in s:
                    if w==word:
                              count+=1
          return count

str1 = input("Enter any sentence :")
word = input("Enter word to search in sentence :")
count = countWord(str1,word)
if count==0:
          print("## Sorry! ",word," not present ")
else:
          print("## ",word," occurs ",count," times ## ")
\end{lstlisting}

\textbf{OUTPUT}
\begin{verbatim}
Enter any sentence :my computer your computer our computer everyones computer

Enter word to search in sentence :computer
##  computer  occurs  4  times ##

Enter any sentence :learning python is fun
Enter word to search in sentence :java
## Sorry!  java  not present
\end{verbatim}



% --------------------------------------------------

\newpage
\section*{Date : \hfill Experiment No: 6}

\textbf{Program 1: Program to read and display file content line by line with each
word separated by ‘’}

%\begin{verbatim}
\begin{lstlisting}[style=pythonstyle]
#Program to read content of file line by line
#and display each word separated by '#'

f = open("file1.txt")

for line in f:
          words = line.split()
          for w in words:
                    print(w+'#',end='')
          print()
f.close()
\end{lstlisting}

NOTE : if the original content of file is:
\begin{verbatim}
India is my country
I love python
Python learning is fun
\end{verbatim}

\textbf{OUTPUT}
\begin{verbatim}
India#is#my#country#
I#love#python#
Python#learning#is#fun#
\end{verbatim}



% --------------------------------------------------

\newpage
\section*{Date : \hfill Experiment No: 7}

\textbf{Program 1: Program to read the content of file and display the total number
of consonants, uppercase, vowels and lower case characters}

%\begin{verbatim}
\begin{lstlisting}[style=pythonstyle]
#Program to read content of file
#and display total number of vowels, consonants, lowercase and uppercase characters
f = open("file1.txt")
v, c, u, l, o = 0, 0, 0, 0
data = f.read()
vowels=['a','e','i','o','u']

for ch in data:
          if ch.isalpha():
                    if ch.lower() in vowels:
                              v+=1
                    else:
                              c+=1
          if ch.isupper():
                    u+=1
          elif ch.islower():
                    l+=1
          elif ch!=' ' and ch!='\n':
                    o+=1
                    
print("Total Vowels in file             :",v)
print("Total Consonants in file    n    :",c)
print("Total Capital letters in file    :",u)
print("Total Small letters in file      :",l)
print("Total Other than letters         :",o)

f.close()
\end{lstlisting}
NOTE : if the original content of file is:
\begin{verbatim}
India is my country
I love python
Python learning is fun
123@
\end{verbatim}
\textbf{OUTPUT}
\begin{verbatim}
Total Vowels in file            : 16
Total Consonants in file        : 30
Total Capital letters in file   : 2
Total Small letters in file     : 44
Total Other than letters        : 4
\end{verbatim}



\newpage
\section*{Date : \hfill Experiment No: 8}

\textbf{Program 1: Program to create binary file to store Rollno and Name, Search
any Rollno and display name if Rollno found otherwise “Rollno not found”}

%\begin{verbatim}
\begin{lstlisting}[style=pythonstyle]
#Program to create a binary file to store Rollno and name
#Search for Rollno and display record if found
#otherwise "Roll no. not found"

import pickle
student=[]
f=open('student.dat','wb')

ans='y'
while ans.lower()=='y':
          roll = int(input("Enter Roll Number :"))
          name = input("Enter Name :")
          student.append([roll,name])
          ans=input("Add More ?(Y)")
pickle.dump(student,f)
f.close()
f=open('student.dat','rb')

student=[]
while True:
          try:
                    student = pickle.load(f)
          except EOFError:
                    break

ans='y'

while ans.lower()=='y':
          found=False
          r = int(input("Enter Roll number to search :"))
          for s in student:
                    if s[0]==r:
                              print("## Name is :",s[1], " ##")
                              found=True
                              break
          if not found:
                    print("####Sorry! Roll number not found ####")
          ans=input("Search more ?(Y) :")
f.close()
\end{lstlisting}
\newpage
\textbf{OUTPUT}
\begin{verbatim}
Enter Roll Number :1
Enter Name :Amit
Add More ?(Y)y

Enter Roll Number :2
Enter Name :Jasbir
Add More ?(Y)y

Enter Roll Number :3
Enter Name :Vikral
Add More ?(Y)n

Enter Roll number to search :2
## Name is : Jasbir  ##
Search more ?(Y) :y

Enter Roll number to search :1
## Name is : Amit  ##
Search more ?(Y) :y

Enter Roll number to search :4
####Sorry! Roll number not found ####
Search more ?(Y) :n
\end{verbatim}



% --------------------------------------------------

\newpage
\section*{Date : \hfill Experiment No: 9}

\textbf{Program 1: Program to create binary file to store Rollno,Name and Marks
and update marks of entered Rollno}

%\begin{verbatim}
\begin{lstlisting}[style=pythonstyle]
# Program to create a binary file to store Rollno, Name and Marks
# and update marks of entered Rollno

import pickle

student = []
f = open('student.dat', 'wb')

ans = 'y'
while ans.lower() == 'y':
    roll = int(input("Enter Roll Number :"))
    name = input("Enter Name :")
    marks = int(input("Enter Marks :"))
    student.append([roll, name, marks])
    ans = input("Add More ?(Y)")

pickle.dump(student, f)
f.close()

# Reading data from binary file
f = open('student.dat', 'rb')
student = []

while True:
    try:
        student = pickle.load(f)
    except EOFError:
        break
f.close()

# Updating record
ans = 'y'
while ans.lower() == 'y':
    found = False
    r = int(input("Enter Roll number to update :"))
    for s in student:
        if s[0] == r:
            print("## Name is :", s[1], " ##")
            print("## Current Marks is :", s[2], " ##")
            m = int(input("Enter new marks :"))
            s[2] = m
            print("## Record Updated ##")
            found = True
            break

    if not found:
        print("####Sorry! Roll number not found ####")

    ans = input("Update more ?(Y) :")

# Writing updated data back to file
f = open('student.dat', 'wb')
pickle.dump(student, f)
f.close()
\end{lstlisting}

\textbf{OUTPUT}
\begin{verbatim}
Enter Roll Number :1
Enter Name :Amit
Enter Marks :99
Add More ?(Y)y

Enter Roll Number :2
Enter Name :Vikrant
Enter Marks :88
Add More ?(Y)y

Enter Roll Number :3
Enter Name :Nitin
Enter Marks :66
Add More ?(Y)n

Enter Roll number to update :2
## Name is : Vikrant  ##
## Current Marks is : 88  ##
Enter new marks :90
## Record Updated ##
Update more ?(Y) :y

Enter Roll number to update :2
## Name is : Vikrant  ##
## Current Marks is : 90  ##
Enter new marks :95
## Record Updated ##
Update more ?(Y) :n
\end{verbatim}



% --------------------------------------------------

\newpage
\section*{Date : \hfill Experiment No: 10}

\textbf{Program 1: Program to read the content of file line by line and write it to
another file except for the lines contains ‘a’ letter in it.}

%\begin{verbatim}
\begin{lstlisting}[style=pythonstyle]
#Program to read line from file and write it to another line
#Except for those line which contains letter 'a'

f1 = open("file2.txt")
f2 = open("file2copy.txt","w")

for line in f1:
          if 'a' not in line:
                    f2.write(line)
print("## File Copied Successfully! ##")

f1.close()
f2.close()
\end{lstlisting}

NOTE: Content of file2.txt
\begin{verbatim}
a quick brown fox
one two three four

five six seven
India is my country

eight nine ten
bye!
\end{verbatim}

\textbf{OUTPUT}
\begin{verbatim}
## File Copied Successfully! ##
\end{verbatim}

NOTE: After copy content of file2copy.txt
\begin{verbatim}
one two three four
five six seven

eight nine ten

bye!
\end{verbatim}



% --------------------------------------------------

\newpage
\section*{Date : \hfill Experiment No: 11}

\textbf{Program 1: Program to create CSV file and store empno,name,salary and
search any empno and display name,salary and if not found appropriate
message.}


%\begin{verbatim}
\begin{lstlisting}[style=pythonstyle]
import csv

# Writing data into CSV file
with open('myfile.csv', mode='a', newline='') as csvfile:
    mywriter = csv.writer(csvfile, delimiter=',')
    ans = 'y'
    while ans.lower() == 'y':
        eno = int(input("Enter Employee Number "))
        name = input("Enter Employee Name ")
        salary = int(input("Enter Employee Salary :"))
        mywriter.writerow([eno, name, salary])
        print("## Data Saved... ##")
        ans = input("Add More ?")

# Searching record from CSV file
ans = 'y'
with open('myfile.csv', mode='r') as csvfile:
    myreader = csv.reader(csvfile, delimiter=',')
    while ans.lower() == 'y':
        found = False
        e = int(input("Enter Employee Number to search :"))
        csvfile.seek(0)
        for row in myreader:
            if len(row) != 0:
                if int(row[0]) == e:
                    print("============================")
                    print("NAME   :", row[1])
                    print("SALARY :", row[2])
                    found = True
                    break
        if not found:
            print("==========================")
            print("      EMPNO NOT FOUND")
            print("==========================")
        ans = input("Search More ? (Y)")
\end{lstlisting}


\textbf{OUTPUT}
\begin{verbatim}
Enter Employee Number 1
Enter Employee Name Amit
Enter Employee Salary :90000
## Data Saved... ##
Add More ?y

Enter Employee Number 2
Enter Employee Name Sunil
Enter Employee Salary :80000
## Data Saved... ##
Add More ?y

Enter Employee Number 3
Enter Employee Name Satya
Enter Employee Salary :75000
## Data Saved... ##
Add More ?n

Enter Employee Number to search :2
============================
NAME   : Sunil
SALARY : 80000
Search More ? (Y)y

Enter Employee Number to search :3
============================
NAME   : Satya
SALARY : 75000
Search More ? (Y)y

Enter Employee Number to search :4
==========================
      EMPNO NOT FOUND
==========================
Search More ? (Y)n
\end{verbatim}



% --------------------------------------------------

\newpage
\section*{Date : \hfill Experiment No: 12}

\textbf{Program 1: Program to generate random number 1-6, simulating a dice}

%\begin{verbatim}
\begin{lstlisting}[style=pythonstyle]
# Program to generate random number between 1 - 6
# To simulate the dice
import random
import time
print("Press CTRL+C to stop the dice ")

play='y'
while play=='y':
          try:
                    while True:
                              for i in range(10):
                                        print()
                              n = random.randint(1,6)
                              print(n,end='')
                              time.sleep(.00001)
          except KeyboardInterrupt:
                    print("Your Number is :",n)
                    ans=input("Play More? (Y) :")
                    if ans.lower()!='y':
                              play='n'
                              break
\end{lstlisting}

\textbf{OUTPUT}
\begin{verbatim}
4Your Number is : 4
Play More? (Y) :y
Your Number is : 3
Play More? (Y) :y
Your Number is : 2
Play More? (Y) :n
\end{verbatim}




\newpage
\section*{Date : \hfill Experiment No: 13}

\textbf{Program 1: Program to implement Stack in Python using List}

%\begin{verbatim}
\begin{lstlisting}[style=pythonstyle]
def isEmpty(S):
 if len(S)==0:
  return True
 else:
  return False

def Push(S,item):
 S.append(item)
 top=len(S)-1

def Pop(S):
 if isEmpty(S):
  return "Underflow"
 else:
  val = S.pop()
  if len(S)==0:
   top=None
  else:
   top=len(S)-1
  return val

def Peek(S):
 if isEmpty(S):
  return "Underflow"
 else:
  top=len(S)-1
  return S[top]

def Show(S):
 if isEmpty(S):
  print("Sorry No items in Stack ")
 else:
  t = len(S)-1
  print("(Top)",end=' ')
  while(t>=0):
   print(S[t],"<==",end=' ')
   t-=1
  print()





\# main begins here
S=[]  \#Stack
top=None

while True:
print("**** STACK DEMONSTRATION ******")
print("1. PUSH ")
print("2. POP")
print("3. PEEK")
print("4. SHOW STACK ")
print("0. EXIT")
ch = int(input("Enter your choice :"))
if ch==1:
 val = int(input("Enter Item to Push :"))
 Push(S,val)

elif ch==2:
 val = Pop(S)
 if val=="Underflow":
  print("Stack is Empty")
 else:
  print("Deleted Item was :",val)

elif ch==3:
 val = Peek(S)
 if val=="Underflow":
  print("Stack Empty")
 else:
  print("Top Item :",val)

elif ch==4:
 Show(S)

elif ch==0:
 print("Bye")
 break

\end{lstlisting}

\textbf{OUTPUT}
\begin{verbatim}
**** STACK DEMONSTRATION ******
1. PUSH
2. POP
3. PEEK
4. SHOW STACK
0. EXIT
Enter your choice :1
Enter Item to Push :10


**** STACK DEMONSTRATION ******
1. PUSH
2. POP
3. PEEK
4. SHOW STACK
0. EXIT
Enter your choice :1
Enter Item to Push :20

**** STACK DEMONSTRATION ******
1. PUSH
2. POP
3. PEEK
4. SHOW STACK
0. EXIT
Enter your choice :1
Enter Item to Push :30

**** STACK DEMONSTRATION ******
1. PUSH
2. POP
3. PEEK
4. SHOW STACK
0. EXIT
Enter your choice :4

(Top) 30 <== 20 <== 10 <==

**** STACK DEMONSTRATION ******
1. PUSH
2. POP
3. PEEK
4. SHOW STACK
0. EXIT
Enter your choice :3
Top Item : 30

**** STACK DEMONSTRATION ******
1. PUSH
2. POP
3. PEEK
4. SHOW STACK
0. EXIT
Enter your choice :2

Deleted Item was : 30


**** STACK DEMONSTRATION ******
1. PUSH
2. POP
3. PEEK
4. SHOW STACK
0. EXIT
Enter your choice :4

(Top) 20 <== 10 <==

**** STACK DEMONSTRATION ******
1. PUSH
2. POP
3. PEEK
4. SHOW STACK
0. EXIT
Enter your choice :0
Bye
\end{verbatim}



% --------------------------------------------------

\newpage
\section*{Date : \hfill Experiment No: 14}

\textbf{Program 1: Program to implement Queue in Python using List}


%\begin{verbatim}
\begin{lstlisting}[style=pythonstyle]
def isEmpty(Q):
    if len(Q) == 0:
        return True
    else:
        return False

def Enqueue(Q, item):
    Q.append(item)
    if len(Q) == 1:
        front = rear = 0
    else:
        rear = len(Q) - 1
def Dequeue(Q):
    if isEmpty(Q):
        return "Underflow"
    else:
        val = Q.pop(0)
        if len(Q) == 0:
            front = rear = None
        return val
def Peek(Q):
    if isEmpty(Q):
        return "Underflow"
    else:
        front = 0
        return Q[front]

def Show(Q):
    if isEmpty(Q):
        print("Sorry No items in Queue ")
    else:
        print("(Front)", end=' ')
        for item in Q:
            print(item, "==>", end=' ')
        print()
Q = []     # Queue
front = rear = None
while True:
    print("**** QUEUE DEMONSTRATION ******")
    print("1. ENQUEUE ")
    print("2. DEQUEUE")
    print("3. PEEK")
    print("4. SHOW QUEUE ")
    print("0. EXIT")
    ch = int(input("Enter your choice :"))

    if ch == 1:
        val = int(input("Enter Item to Insert :"))
        Enqueue(Q, val)
    elif ch == 2:
        val = Dequeue(Q)
        if val == "Underflow":
            print("Queue is Empty")
        else:
            print("\nDeleted Item was :", val)
    elif ch == 3:
        val = Peek(Q)
        if val == "Underflow":
            print("Queue Empty")
        else:
            print("Front Item :", val)

    elif ch == 4:
        Show(Q)

    elif ch == 0:
        print("Bye")
        break
\end{lstlisting}



\textbf{OUTPUT}
\begin{verbatim}
**** QUEUE DEMONSTRATION ******
1. ENQUEUE
2. DEQUEUE
3. PEEK
4. SHOW QUEUE
0. EXIT
Enter your choice :1
Enter Item to Insert :10

**** QUEUE DEMONSTRATION ******
1. ENQUEUE
2. DEQUEUE
3. PEEK
4. SHOW QUEUE
0. EXIT
Enter your choice :1
Enter Item to Insert :30

**** QUEUE DEMONSTRATION ******
1. ENQUEUE
2. DEQUEUE
3. PEEK
4. SHOW QUEUE
0. EXIT
Enter your choice :4
(Front) 10 ==> 20 ==> 30 ==>

**** QUEUE DEMONSTRATION ******
1. ENQUEUE
2. DEQUEUE
3. PEEK
4. SHOW QUEUE
0. EXIT
Enter your choice :3
Front Item : 10

**** QUEUE DEMONSTRATION ******
1. ENQUEUE
2. DEQUEUE
3. PEEK
4. SHOW QUEUE
0. EXIT
Enter your choice :2
Deleted Item was : 10

**** QUEUE DEMONSTRATION ******
1. ENQUEUE
2. DEQUEUE
3. PEEK
4. SHOW QUEUE
0. EXIT
Enter your choice :4
(Front) 20 ==> 30 ==>

**** QUEUE DEMONSTRATION ******
1. ENQUEUE
2. DEQUEUE
3. PEEK
4. SHOW QUEUE
0. EXIT
Enter your choice :0
Bye
\end{verbatim}







\newpage
\section*{Date : \hfill Experiment No: 15}

\textbf{Program 1: Program to take 10 sample phishing email, and find the most
common word occurring}

%\begin{verbatim}
\begin{lstlisting}[style=pythonstyle]
#Program to take 10 sample phishing mail
#and count the most commonly occuring word
phishingemail=[
          "jackpotwin@lottery.com",
          "claimtheprize@mymoney.com",
          "youarethewinner@lottery.com",
          "luckywinner@mymoney.com",
          "spinthewheel@flipkart.com",
          "dealwinner@snapdeal.com"
          "luckywinner@snapdeal.com"
          "luckyjackpot@americanlottery.com"
          "claimtheprize@lootolottery.com"
          "youarelucky@mymoney.com"
          ]
myd={}
for e in phishingemail:
          x=e.split('@')
          for w in x:
                    if w not in myd:
                              myd[w]=1
                    else:
                              myd[w]+=1

key_max = max(myd,key=myd.get)
print("Most Common Occuring word is :",key_max)
\end{lstlisting}

\textbf{OUTPUT}
\begin{verbatim}
Most Common Occuring word is : mymoney.com
\end{verbatim}



% --------------------------------------------------

\newpage
\section*{Date : \hfill Experiment No: 16}

\textbf{Program 1: Program to create 21 Stick Game so that computer always wins}


Rule of Game (Total Sticks = 21):
1) User and Computer both can pick stick one by one
2) Maximum stick both can pick is 4 i.e. 1 to 4
3) Anyone with last stick will be the looser

%\begin{verbatim}
\begin{lstlisting}[style=pythonstyle]
def PrintStick(n):
    print("o " * n)
    print("| " * n)
    print("| " * n)
    print("| " * n)
    print("| " * n)
TotalStick = 21
win = False
humanPlayer = True
print("=== Welcome To Stick Picking Game :: Computer Vs User ===")
print("Rule: 1) Both User and Computer can pick sticks between 1 to 4 at a time")
print("      2) Whosoever picks the last stick will be the looser")
print("==== Lets Begin ======")
playerName = input("Enter Your Name :")
userPick = 0
PrintStick(TotalStick)
while win == False:
    if humanPlayer == True:
        print("You Can Pick stick between 1 to 4")
        userPick = 0
        while userPick <= 0 or userPick > 4:
            userPick = int(input(playerName + ": Enter Number of Stick to Pick"))
        TotalStick = TotalStick - userPick
        humanPlayer = False
        PrintStick(TotalStick)
        print("*" * 60)
        input("Press any key...")
    else:
        computerPick = (5 - userPick)
        print("Computer Picks : ", computerPick, " Sticks ")
        TotalStick = TotalStick - computerPick
        PrintStick(TotalStick)
        if TotalStick == 1:
            print("## WINNER : COMPUTER ##")
            win = True
        print("*" * 60)
        input("Press any key...")
        humanPlayer = True
\end{lstlisting}


\textbf{OUTPUT}
\begin{verbatim}
=== Welcome To Stick Picking Game :: Computer Vs User ===
Rule: 1) Both User and Computer can pick sticks between 1 to 4 at a time
      2) Whosoever picks the last stick will be the looser
==== Lets Begin ======
Enter Your Name :RAJ
o o o o o o o o o o o o o o o o o o o o o
| | | | | | | | | | | | | | | | | | | | |
| | | | | | | | | | | | | | | | | | | | |
| | | | | | | | | | | | | | | | | | | | |
| | | | | | | | | | | | | | | | | | | | |

You Can Pick stick between 1 to 4
RAJ: Enter Number of Stick to Pick3
o o o o o o o o o o o o o o o o o o
| | | | | | | | | | | | | | | | | |
| | | | | | | | | | | | | | | | | |
| | | | | | | | | | | | | | | | | |
| | | | | | | | | | | | | | | | | |
************************************************************
Press any key...

Computer Picks :  2  Sticks
o o o o o o o o o o o o o o o
| | | | | | | | | | | | | | |
| | | | | | | | | | | | | | |
| | | | | | | | | | | | | | |
| | | | | | | | | | | | | | |
************************************************************
Press any key...

You Can Pick stick between 1 to 4
RAJ: Enter Number of Stick to Pick4
o o o o o o o o o o o
| | | | | | | | | | |
| | | | | | | | | | |
| | | | | | | | | | |
| | | | | | | | | | |
************************************************************
Press any key...

Computer Picks :  1  Sticks
o o o o o o o o o o
| | | | | | | | | |
| | | | | | | | | |
| | | | | | | | | |
| | | | | | | | | |
************************************************************
Press any key...
You Can Pick stick between 1 to 4
RAJ: Enter Number of Stick to Pick2
o o o o o o o o
| | | | | | | |
| | | | | | | |
| | | | | | | |
| | | | | | | |
************************************************************
Press any key...

Computer Picks :  3  Sticks
o o o o o
| | | | |
| | | | |
| | | | |
| | | | |
************************************************************
Press any key...

You Can Pick stick between 1 to 4
RAJ: Enter Number of Stick to Pick3
o o
| |
| |
| |
| |
************************************************************
Press any key...

Computer Picks :  2  Sticks
o
| 
| 
| 
| 
## WINNER : COMPUTER ##
************************************************************
Press any key...
\end{verbatim}






% --------------------------------------------------

\newpage
\section*{Date : \hfill Experiment No: 17}

\textbf{Program 1: Program to connect with database and store record of employee
and display records.}


%\begin{verbatim}
\begin{lstlisting}[style=pythonstyle]
import mysql.connector as mycon
# Establish connection
con = mycon.connect(
    host='127.0.0.1',
    user='root',
    password='admin'
)

cur = con.cursor()
# Create database and table
cur.execute("create database if not exists company")
cur.execute("use company")
cur.execute(
    "create table if not exists employee("
    "empno int, "
    "name varchar(20), "
    "dept varchar(20), "
    "salary int)"
)
con.commit()
choice = None
while choice != 0:
    print("1. ADD RECORD ")
    print("2. DISPLAY RECORD ")
    print("0. EXIT")
    choice = int(input("Enter Choice :"))
    if choice == 1:
        e = int(input("Enter Employee Number :"))
        n = input("Enter Name :")
        d = input("Enter Department :")
        s = int(input("Enter Salary :"))
        query = "insert into employee values({},'{}','{}',{})".format(e, n, d, s)
        cur.execute(query)
        con.commit()
        print("## Data Saved ##")
    elif choice == 2:
        query = "select * from employee"
        cur.execute(query)
        result = cur.fetchall()
        print("%10s" % "EMPNO", "%20s" % "NAME",
              "%15s" % "DEPARTMENT", "%10s" % "SALARY")
        for row in result:
            print("%10s" % row[0], "%20s" % row[1],
                  "%15s" % row[2], "%10s" % row[3])

    elif choice == 0:
        con.close()
        print("## Bye!! ##")
    else:
        print("## INVALID CHOICE ##")
\end{lstlisting}


\textbf{OUTPUT}
\begin{verbatim}
1. ADD RECORD
2. DISPLAY RECORD
0. EXIT
Enter Choice :1
Enter Employee Number :1
Enter Name :AMIT
Enter Department :SALES
Enter Salary :9000
## Data Saved ##

1. ADD RECORD
2. DISPLAY RECORD
0. EXIT
Enter Choice :1
Enter Employee Number :2
Enter Name :NITIN
Enter Department :IT
Enter Salary :80000
## Data Saved ##

1. ADD RECORD
2. DISPLAY RECORD
0. EXIT
Enter Choice :2
     EMPNO                 NAME      DEPARTMENT     SALARY
         1                 AMIT           SALES       9000
         2                NITIN              IT      80000

1. ADD RECORD
2. DISPLAY RECORD
0. EXIT
Enter Choice :0
## Bye!! ##
\end{verbatim}




\newpage
\section*{Date : \hfill Experiment No: 18}

\textbf{Program 1: Program to connect with database and search employee number
in table employee and display record, if empno not found display
appropriate message.}



%\begin{verbatim}
\begin{lstlisting}[style=pythonstyle]
import mysql.connector as mycon
con = mycon.connect(host='127.0.0.1', user='root', password='admin', database='company')
cur = con.cursor()
print("#" * 40)
print("EMPLOYEE SEARCHING FORM")
print("#" * 40, end="\n\n")
ans = 'y'
while ans.lower() == 'y':
    eno = int(input("ENTER EMPNO TO SEARCH :"))
    query = "select * from employee where empno={}".format(eno)
    cur.execute(query)
    result = cur.fetchall()
    if cur.rowcount == 0:
        print("Sorry! Empno not found ")
    else:
        print("%10s" % "EMPNO", "%20s" % "NAME",
              "%15s" % "DEPARTMENT", "%10s" % "SALARY")
        for row in result:
            print("%10s" % row[0], "%20s" % row[1],
                  "%15s" % row[2], "%10s" % row[3])
    ans = input("SEARCH MORE (Y) :")
con.close()
\end{lstlisting}


\textbf{OUTPUT}
\begin{verbatim}
########################################
EMPLOYEE SEARCHING FORM
########################################

ENTER EMPNO TO SEARCH :1
     EMPNO                 NAME      DEPARTMENT     SALARY
         1                 AMIT           SALES       9000
SEARCH MORE (Y) :y
ENTER EMPNO TO SEARCH :2
     EMPNO                 NAME      DEPARTMENT     SALARY
         2                NITIN              IT      80000
SEARCH MORE (Y) :y
ENTER EMPNO TO SEARCH :4
Sorry! Empno not found
SEARCH MORE (Y) :n
\end{verbatim}



% --------------------------------------------------

\newpage
\section*{Date : \hfill Experiment No: 19}

\textbf{Program 1: Program to connect with database and update the employee
record of entered empno.}

%\begin{verbatim}
\begin{lstlisting}[style=pythonstyle]
import mysql.connector as mycon
con = mycon.connect(host='127.0.0.1',user='root',password="admin",
database="company")
cur = con.cursor()
print("#"*40)
print("EMPLOYEE UPDATION FORM")
print("#"*40)
print("\n\n")
ans='y'
while ans.lower()=='y':
 eno = int(input("ENTER EMPNO TO UPDATE :"))
 query="select * from employee where empno={}".format(eno)
 cur.execute(query)
 result = cur.fetchall()
 if cur.rowcount==0:
  print("Sorry! Empno not found ")
 else: print("%10s"%"EMPNO","%20s"%"NAME", "%15s"%"DEPARTMENT",
"%10s"%"SALARY")
  for row in result: print("%10s"%row[0],"%20s"%row[1],"%15s"%row[2],"%10s"%row[3])
  choice=input("\n## ARE YOUR SURE TO UPDATE ? (Y) :")
  if choice.lower()=='y':
   print("== YOU CAN UPDATE ONLY DEPT AND SALARY ==")
   print("== FOR EMPNO AND NAME CONTACT ADMIN ==")
   d = input("ENTER NEW DEPARTMENT,(LEAVE BLANK IF NOT WANT TO CHANGE )")
   if d=="":
    d=row[2]
   try:
    s = int(input("ENTER NEW SALARY,(LEAVE BLANK IF NOT WANT TO CHANGE ) "))
   except:
    s=row[3]
   query="update employee set dept='{}',salary={} where empno={}".format(d,s,eno)
   cur.execute(query)
   con.commit()
   print("## RECORD UPDATED ## ")
 ans=input("UPDATE MORE (Y) :")
\end{lstlisting}

\textbf{OUTPUT}
\begin{verbatim}
########################################
EMPLOYEE UPDATION FORM
########################################

ENTER EMPNO TO UPDATE :2
     EMPNO                 NAME      DEPARTMENT     SALARY
         2                NITIN              IT      90000

## ARE YOUR SURE TO UPDATE ? (Y) :y
== YOU CAN UPDATE ONLY DEPT AND SALARY ==
== FOR EMPNO AND NAME CONTACT ADMIN ==
ENTER NEW DEPARTMENT,(LEAVE BLANK IF NOT WANT TO CHANGE )
ENTER NEW SALARY,(LEAVE BLANK IF NOT WANT TO CHANGE )
## RECORD UPDATED ##

UPDATE MORE (Y) :y
ENTER EMPNO TO UPDATE :2
     EMPNO                 NAME      DEPARTMENT     SALARY
         2                NITIN              IT      90000

## ARE YOUR SURE TO UPDATE ? (Y) :y
== YOU CAN UPDATE ONLY DEPT AND SALARY ==
== FOR EMPNO AND NAME CONTACT ADMIN ==
ENTER NEW DEPARTMENT,(LEAVE BLANK IF NOT WANT TO CHANGE )SALES
ENTER NEW SALARY,(LEAVE BLANK IF NOT WANT TO CHANGE )
## RECORD UPDATED ##
UPDATE MORE (Y) :Y
ENTER EMPNO TO UPDATE :2
     EMPNO                 NAME      DEPARTMENT     SALARY
         2                NITIN           SALES      90000

## ARE YOUR SURE TO UPDATE ? (Y) :Y
== YOU CAN UPDATE ONLY DEPT AND SALARY ==
== FOR EMPNO AND NAME CONTACT ADMIN ==
ENTER NEW DEPARTMENT,(LEAVE BLANK IF NOT WANT TO CHANGE )
ENTER NEW SALARY,(LEAVE BLANK IF NOT WANT TO CHANGE ) 91000
## RECORD UPDATED ##
UPDATE MORE (Y) :Y

ENTER EMPNO TO UPDATE :2
     EMPNO                 NAME      DEPARTMENT     SALARY
         2                NITIN           SALES      91000

## ARE YOUR SURE TO UPDATE ? (Y) :N
UPDATE MORE (Y) :N
\end{verbatim}



% --------------------------------------------------

\newpage
\section*{Date : \hfill Experiment No: 20}

\textbf{Program 1: Program to connect with database and delete the record of
entered employee number.}

%\begin{verbatim}
\begin{lstlisting}[style=pythonstyle]
import mysql.connector as mycon
con = mycon.connect(host='127.0.0.1',user='root',password="admin",
database="company")

cur = con.cursor()
print("#"*40)
print("EMPLOYEE DELETION FORM")
print("#"*40)
print("\n\n")

ans='y'
while ans.lower()=='y':

 eno = int(input("ENTER EMPNO TO DELETE :"))
 query="select * from employee where empno={}".format(eno)
 cur.execute(query)
 result = cur.fetchall()

 if cur.rowcount==0:
  print("Sorry! Empno not found ")
 else:
  print("%10s"%"EMPNO","%20s"%"NAME", "%15s"%"DEPARTMENT",
"%10s"%"SALARY")
  for row in result:
   print("%10s"%row[0],"%20s"%row[1],"%15s"%row[2],"%10s"%row[3])

  choice=input("\n## ARE YOUR SURE TO DELETE ? (Y) :")
  if choice.lower()=='y':
   query="delete from employee where empno={}".format(eno)
   cur.execute(query)
   con.commit()
   print("=== RECORD DELETED SUCCESSFULLY! ===")

 ans=input("DELETE MORE ? (Y) :")
\end{lstlisting}

\textbf{OUTPUT}
\begin{verbatim}
########################################
EMPLOYEE DELETION FORM
########################################

ENTER EMPNO TO DELETE :2
 EMPNO  NAME  DEPARTMENT  SALARY
 2  NITIN  SALES  91000

## ARE YOUR SURE TO DELETE ? (Y) :y
=== RECORD DELETED SUCCESSFULLY! ===
DELETE MORE ? (Y) :y
ENTER EMPNO TO DELETE :2
Sorry! Empno not found
DELETE MORE ? (Y) :n
\end{verbatim}





% -------------------------------------------------
% The same structure continues for ALL experiments
% EXACT text, spacing, outputs, notes, and pages
% -------------------------------------------------

% \newpage
% \begin{center}
% \textbf{python4csip.com}
% \end{center}

\end{document}
